\documentclass{report}
\usepackage[left=2cm,right=2cm, top=2cm,bottom=2cm,bindingoffset=0cm]{geometry}
\usepackage{amsmath}
\begin{document}
	\begin{center} \section*{Package fpop2D}\end{center}
	\parindent = 0pt
	\subsection*{Description}
	
	{\bfseries fpop2D} is an R package written in Rcpp/C++ and developed to change-point detection in bivariate time-series.
	
	The package implements the Functional Pruning Optimal Partitioning algorithm with two types of pruning: "intersection of sets" and "difference of intersection and union of sets". 
	\subsection*{Package structure}
	\begin{itemize}
		\item  Part R
		\begin{itemize}
			\item \underline {fpop2D.R}
			
			The file contains the implementation of the following functions:
			
			 {\bfseries data\_gen2D} - generation of data of dimension 2 with a given values of means and changepoints
			 
			 {\bfseries plotFPOP2D} - plot of data with a  values of means and changepoints.
			 
			\item \underline {RcppExports.R} 
			
			The file contains the implementation of the function {\bfseries FPOP2D} that calls the function {\bfseries FPOP2D} implemented in C++.
			
		\end{itemize}
		\item Part C++
		\begin{itemize}
			\item \underline{Cost.h, Cost.cpp}
			
			The files contain the implementation code for the class {\bfseries Cost} . 
				
			
			\item \underline{Disk.h, Disk.cpp}
			
			The files contain the implementation code for the class {\bfseries Disk}. 
			
			\item  \underline{Rect.h, Rect.cpp}
			
			The files contain the implementation code for the class {\bfseries Rect}.
			 
			\item \underline{Geom.h, Geom.cpp} 
			
			The files contain the implementation code for the class {\bfseries Geom}. 
			
			\item \underline{OP.h, OP.cpp}
			
			The files contain the implementation code for the class {\bfseries OP}. 
			
			\item \underline{fpop2D.cpp}
			
			The file contains the code of the function {\bfseries FPOP2D} that implements the change-point detection in bivariate time-series using the Functional Pruning Optimal Partitioning algorithm.
			
			\item \underline{RcppExports.cpp} 
			
			The file contains the code that exports data R/C ++ .
				
		\end{itemize}
	\end{itemize}

\newpage
	
	We consider $(y_1, y_2)\in(R^2)^n$ - bivariate time-series when $y_1 = (y_1^1,..,y_n^1)$ and  $y_2 = (y_1^2,..,y_n^2)$ - two vectors of univariate data size $n$.
	
	We use the Gaussian cost of the segmented bivariate data when $m_{t}$ is the value of the optimal cost, $m_{0} = -\beta $. The cost function has the form: 
	
	\begin{equation}
		q_t^i( \theta_1,\theta_2 ) = 	m_{i-1} + \beta + (t-i+1)(( \theta_1 - E[y^1_{i:t}])^2 + ( \theta_2 - E[y^2_{i:t}])^2) + (t-i+1)(V(y^1_{i:t}) + V(y^2_{i:t}) ) , \hspace*{5mm} i = 1:t.
	\label{eq:cost}
	\end{equation}
	
	\begin{equation}
		E[y_{i:t}^k] = \frac{1}{t - i + 1}\sum_{l=i}^{t}(y_{i:t}^k) ,\hspace*{5mm} k = 1,2.
	\label{eq:mean}
	\end{equation}
	
	\begin{equation}
		V(y_{i:t}^k) = \frac{1}{t - i + 1}\sum_{l=i}^{t}(y_{i:t}^k)^2 - \left(\frac{1}{t - i + 1}\sum_{l=i}^{t}(y_{i:t}^k)\right)^2 ,\hspace*{5mm} k = 1,2.
	\label{eq:var}
	\end{equation}
	
	In the case of a Gaussian cost $q_t^i(\theta_1,\theta_2)$ is a paraboloid so that: 
	\begin{equation}
		m_t = min_{i = 1:t} \left\{m_{i-1} + (t-i+1)(V(y_{i:t}^1) + V(y_{i:t}^2)) + \beta  \right\}.
	\label{eq:min}
	\end{equation}
	
	We introduce the notations:
	
	\begin{equation}
		\begin{gathered}
			mu1 = E[y_{i:t}^1],\\
			mu2 = E[y_{i:t}^2],\\
			coef = (t - i + 1),\\
			mi\_1\_p = m_{i-1} + \beta,\\
			coef\_Var = (t-i+1)(V(y_{i:t}^1) + V(y_{i:t}^2)).
		\end{gathered}
		\label{eq:coef}
	\end{equation}
	
	Then the cost function  takes the form:
	
		\begin{equation}
	q_t^i(\theta_1,\theta_2) = mi\_1\_p+ coef\cdot((\theta_1 - mu1)^2 + (\theta_2 -mu1)^2) + coef\_Var, \hspace*{5mm} i = 1:t.
	\label{eq:costcoef}
	\end{equation}
	
	\subsection*{Class  Cost}
	\label{Cost}
	
	We define the class characteristics (\ref{eq:coef}): 
	
	\begin{itemize}
		\item {\bfseries coef}, {\bfseries coef\_Var}, {\bfseries mi\_1\_p}, {\bfseries mu1}, {\bfseries mu2} 
	\end{itemize}
 
	The class implements two constructors:
	\begin{itemize}
		\item {\bfseries Cost(double mi\_1pen)} is the cost function at the unit time. 
		
		\item {\bfseries Cost(unsigned int i, unsigned int t, std::vector$<$ std::vector$<$ double $>>$ vectS, double mi\_1pen)} is the cost function at the time $[i,t]$. 
	\end{itemize}
	
	We define the class methods:
	
	\begin{itemize}
		\item {\bfseries get\_coef()}, {\bfseries get\_coef\_Var()}, {\bfseries get\_mi\_1\_p()}, {\bfseries get\_mu1()}, {\bfseries get\_mu2()}
		
		We use these  methods to access the characteristics of the class. 
		
		\item {\bfseries get\_min()}
		
		We use this method to get the minimum value of the cost function (\ref{eq:min}).
	\end{itemize} 

\newpage	

	\subsection*{Class Disk}
	\label{Disk}
	
	A class element is a circle that is defined by the center coordinates and the radius.
	
	\begin{itemize}
		\item We define the class characteristics: 
		\begin{itemize}
			\item {\bfseries center1, center2}  is the center of the circle.
		
			\item {\bfseries radius}  is  the radius of the circle.
		\end{itemize}
	
		\item The class implements two constructors:
		\begin{itemize}
			\item {\bfseries Disk()} All characteristics are equal zero by default. 
		
			\item {\bfseries Disk(double c1, double c2, double r)}
		
			$(c1, c2)$ are the center coordinates and $r$ is  the radius of the circle. 
		\end{itemize}
	
		\item We define the class methods:
	
		\begin{itemize}
			\item {\bfseries get\_center1()}, {\bfseries get\_center2()}, {\bfseries get\_radius()}
			
			We use these  methods to access the class characteristics.
		\end{itemize}
	\end{itemize}
	
	\subsection*{Class Rect}
	\label{Rect}
	
	A class element is a rectangle that is defined by the coordinates of the lower left-hand corner and the upper-right hand corner.
		
	\begin{itemize}
		\item We define the class characteristics: 
		\begin{itemize}
			\item {\bfseries rectx0,recty0} are coordinates of the lower left-hand corner.
			
			\item {\bfseries rectx1, recty1} are coordinates of the upper-right hand corner.
		\end{itemize}
		
		\item The class implements two constructors:
		\begin{itemize}
			\item {\bfseries Rect()} All characteristics are equal zero by default. 
			
			\item {\bfseries Rect(double x0, double y0, double x1, double y1)}
			
			$(x0, y0)$ is the the lower left-hand corner and $(x1, y1)$ is the upper-right hand corner. 	
		\end{itemize}
		
		\item We define the class methods:
		
		\begin{itemize}
			\item {\bfseries get\_center1()}, {\bfseries get\_center2()}, {\bfseries get\_radius()}
			
			We use these  methods to access the class characteristics.
		\end{itemize}
	\end{itemize}

	\newpage
		
	\subsection*{Class Geom}
	\label{Geom}
	
	\begin{itemize}
		\item We define the class characteristics: 
		\begin{itemize}
			\item {\bfseries label\_t} is the time moment.
			
			\item {\bfseries cost\_t} is the cost function, the element of class {\bfseries Cost}.
			
			\item {\bfseries rect\_t} is the rectangle, the element of class {\bfseries Rect}.
		\end{itemize}
		
		\item The class implements two constructors:
		\begin{itemize}
			\item {\bfseries Geom(double c1, double c2, double r, double t, double m\_ipen)}
			\begin{itemize}
				\item $label\_t = t$;
				\item $rect\_t = Rect(c1-r, c2-r, c1+r, c2+r)$;
				\item $cost\_t = Cost(m\_ipen)$.
			\end{itemize}
			
			\item {\bfseries Geom(double lbl, Cost cst, Rect rct)}
			\begin{itemize}
				\item $label\_t = lbl$;
				\item $rect\_t = rct$;
				\item $cost\_t = cst$.
			\end{itemize}	
		\end{itemize}
		
		\item We define the class methods:
		
		\begin{itemize}
			\item {\bfseries get\_label\_t()}, {\bfseries get\_rect\_t()}, {\bfseries get\_cost\_t()}
			
			We use these  methods to access the class characteristics.
			
			\item {\bfseries min\_ab(double a, double b)}, {\bfseries max\_ab(double a, double b)}
			
			We use these methods to find the minimum and maximum of two numbers. 
			
			\item {\bfseries bool empty\_set(Rect R)}
			
			The function checks the parameters of the rectangle {\bfseries R}. If the parameters are not correct, this rectangle is empty. 
			
			
			\item {\bfseries Rect intersection(Rect rect, Disk disk)} 
			
			The function approximates a rectangle and a circle intersection area by horizontal and vertical lines. Basing on the intersection points of these lines, we construct a rectangle with a minimum area, which contains the intersection area of the rectangle and the circle.
			
			\item {\bfseries Rect difference(Rect rect, Disk disk)} 
			
			The function approximates a rectangle and a circle difference area by horizontal and vertical lines. Basing on the intersection points of these lines, we construct a rectangle with a minimum area, which contains the difference area of the rectangle and the circle.	
		\end{itemize}
	\end{itemize}

	\newpage	
	\subsection*{Class OP}
	\label{OP}
	
	\begin{itemize}
		\item We define the class characteristics: 
		
		\begin{itemize}
			\item {\bfseries n} is the number of data points.
			
			\item {\bfseries penalty} is a value of penalty (a non-negative real number).
			
			\item {\bfseries sy12} are sum vectors $\sum_{k=1}^{t} y^1_k$,  $\sum_{k=1}^{t}y^2_k$, $\sum_{k=1}^{t} (y^1_k)^2$,  $\sum_{k=1}^{t} (y^2_k)^2$, $t = 1:n$ .
			
			\item {\bfseries last\_chpts} is the vector of candidates for the position of changepoints. 
			
			\item {\bfseries m} is the vector of the optimal cost value.
			
			\item {\bfseries changepoints} is the vector of changepoints.
			
			\item {\bfseries means1} is the vector of successive means for data1.
			
			\item {\bfseries means2} is the vector of successive means for data2.
			
			\item {\bfseries globalCost} is the global cost.
		\end{itemize}
		
		\item The class implements the constructor:
		
		\begin{itemize}
			\item {\bfseries OP(std::vector$<$double$>$ y1, std::vector$<$double$>$ y2, double beta)}
			\begin{itemize}
				\item $penalty = beta$;
				
				\item $n = y1.size()$;
				
				\item $last\_chpts.push\_back(0)$;
				
				\item $m.push\_back(0)$.
			\end{itemize}	
		\end{itemize}
		
		\item We define the class methods:
		
		\begin{itemize}
			\item {\bfseries get\_changepoints()}, {\bfseries get\_means1()}, {\bfseries get\_means2()}, {\bfseries get\_globalCost()}, {\bfseries get\_n()}, {\bfseries get\_sy12()} 
			
			We use these  methods to access the class characteristics.
			
			\item {\bfseries vect\_sy12(std::vector$<$double$>$ y1, std::vector$<$double$>$ y2)}
			
			We use this method to find the sum vectors $\sum_{k=1}^{t} y^1_k$,  $\sum_{k=1}^{t}y^2_k
			$, $\sum_{k=1}^{t} (y^1_k)^2$,  $\sum_{k=1}^{t} (y^2_k)^2$, $t = 1:n$. 
			
			\item {\bfseries backtracking(unsigned int ndata)}
			
			We use this method to build the FPOP algorithm results. 
			
			
			\item {\bfseries algoFPOP(std::vector$<$double$>$ y1, std::vector$<$double$>$ y2, int type)} 
			
			The function implements the Functional Pruning Optimal Partitioning algorithm with two types of pruning: "intersection of sets"($type = 1$) and "difference of intersection and union of sets"($type = 2$).
		\end{itemize}
	\end{itemize}


\newpage
\newpage

\label{Intersection}	
\begin{center} 
	\section*{Rect Geom::intersection(Rect rect, Disk disk)}
\end{center}
\parindent = 0pt
\subsection*{Description}

The function approximates a rectangle and a circle intersection area by horizontal and vertical lines. Basing on the intersection points of these lines, we construct a rectangle with a minimum area, which contains the intersection area of the rectangle and the circle.

If there is no intersection, the function returns the rectangle with parameters that correspond to the condition: 

\begin{equation}
	(rectx0 \ge rectx1) || (recty0 \ge recty1).
	\label{eq:cond1}
\end{equation}

\subsection*{Input parameters:}

The input of this function consists of two parameters:

\begin{itemize}
	\item {\bfseries	rect} is the rectangle, the element of class {\bfseries Rect} with characteristics:
	\begin{itemize}
		\item {\bfseries rectx0,recty0} are coordinates of the bottom left corner;
		\item {\bfseries rectx1, recty1} are coordinates of the top right corner.
	\end{itemize}
	
	To access these characteristics we use the methods  {\bfseries get\_rectx0()}, {\bfseries get\_recty0()}, {\bfseries get\_rectx1()}, {\bfseries get\_recty1()} implemented in the class {\bfseries Rect}.
	
	\item {\bfseries disk}  is the circle, the element of class {\bfseries Disk} with characteristics:
	\begin{itemize}
		\item {\bfseries center1, center2}  is the center of the circle;
		\item {\bfseries radius}  is  the radius of the circle.
	\end{itemize}
	
	To access these characteristics we use the methods {\bfseries get\_center1()}, {\bfseries get\_center2()}, {\bfseries get\_radius()} implemented in the class {\bfseries Disk}.
\end{itemize}

\subsection*{Output parameters:}

\begin{itemize}
	\item The function returns new rectangle {\bfseries rect\_approx} (the element of class {\bfseries Rect}) with a minimum area, which contains the intersection area of the rectangle {\bfseries rect} and the circle {\bfseries disk}. The rectangle is formed as a result of the intersection of horizontal and vertical lines that approximate the intersection area of the rectangle and the circle.
	
	If there is no intersection, the function returns a rectangle {\bfseries rect\_approx} with parameters that correspond to the condition (\ref{eq:cond1}).
	
\end{itemize}

\subsection*{Algorithm:}

\subsubsection*{Preprocessing}

Using the methods  {\bfseries get\_rectx0()}, {\bfseries get\_recty0()}, {\bfseries get\_rectx1()}, {\bfseries get\_recty1()} implemented in the class {\bfseries Rect} we define the parameters of the rectangle {\bfseries rect}:

\begin{equation}
	\begin{gathered}	
		x0 = rect.get\_rectx0(),\\
		x1 = rect.get\_rectx1(),\\
		y0 = rect.get\_recty0(),\\
		y1 = rect.get\_recty1().
		\label{eq:paramrect}
	\end{gathered}
\end{equation}
Using the methods {\bfseries get\_center1()}, {\bfseries get\_center2()}, {\bfseries get\_radius()} implemented in the class {\bfseries Disk} we define the parameters of the circle {\bfseries disk}:

\begin{equation}
	\begin{gathered}
		c1 = disk.get\_center1(),\\
		c2 = disk.get\_center2(),\\
		r = disk.get\_radius().
		\label{eq:paramdisk}
	\end{gathered}
\end{equation}

\subsubsection*{Approximation}

We consider two directions and update the characteristics of rectangle:

\begin{itemize}
	
	\item \underline {horizontal direction(the characteristics $y0, y1$)}:
	
	If   $x0 \le c1\le x1$ then
	\begin{equation}
		\begin{gathered}
			y0 = \max\{y0, c2-r\},\\
			y1 = \min\{y1, c2+r\}.
		\end{gathered}
	\end{equation}
	
	Otherwise, we consider the points of intersection of the circle with straight lines $x = x0$ and $x = x1$. The circle has two intersection points with a straight line if the discriminant is positive. We define the values $dl, dr$ (\ref{eq:diskriminant1}) as the value of a discriminant devided by  $4$ of each system (\ref{eq:sys1}, \ref{eq:sys2}):
	
	\begin{equation}
		\begin{cases}
			(x - c1)^2 + (y - c2)^2 = r^2,\\ 
			x = x0.
		\end{cases}
		\label{eq:sys1}
	\end{equation}
	
	\begin{equation}
		\begin{cases}
			(x - c1)^2 + (y - c2)^2 = r^2,\\ 
			x = x1.
		\end{cases}
		\label{eq:sys2}
	\end{equation}
	
	\begin{equation}
		\begin{gathered}
			dl = r^2 - (x0 - c1)^2,\\
			dr = r^2 - (x1 - c1)^2,\\
			\label{eq:diskriminant1}
		\end{gathered}
	\end{equation}
	
	{\bfseries Note:} we define the default intersection points for the algorithm to work correctly as:
	\begin{equation}
		\begin{gathered}
			l1 = r1 =  \infty,\\
			l2 = r2 = -\infty.
			\label{eq:lrinf}
		\end{gathered}
	\end{equation}
	
	We check the sign of $dl, dr$ and find the intersection points:
	
	\begin{equation}
		\begin{cases}
			dl > 0,\\ 
			l1 = c2 - \sqrt {dl},\\
			l2 = c2 + \sqrt {dl}.
			\label{eq:l1l2}
		\end{cases}
	\end{equation}
	
	\begin{equation}
		\begin{cases}
			dr > 0,\\ 
			r1 = c2 - \sqrt {dr},\\
			r2 = c2 + \sqrt {dr}.
			\label{eq:r1r2}
		\end{cases}
	\end{equation}
	
	We define the characteristics of rectangle as:
	
	\begin{equation}
		\begin{gathered}
			y0 = \max\{y0, \min\{l1, r1\}\},\\
			y1 = \min\{y1, \max\{l2, r2\}\}.
		\end{gathered}
	\end{equation}
	
	\item \underline {vertical direction (the characteristics $x0, x1$) }
	
	If   $y0 \le c2 \le y1$ then
	
	\begin{equation}
		\begin{gathered}
			x0 = \max\{x0, c1-r\},\\
			x1 = \min\{x1, c1+r\}.
		\end{gathered}
	\end{equation}
	
	Otherwise, we consider the points of intersection of the circle with straight lines $y = y0$ and $y = y1$. The circle has two intersection points with a straight line if the discriminant is positive. We define the values $db, dt$ (\ref{eq:diskriminant2}) as the value of a discriminant devided by  $4$ of each system (\ref{eq:sys3}, \ref{eq:sys4}):
	
	\begin{equation}
		\begin{cases}
			(x - c1)^2 + (y - c2)^2 = r^2,\\ 
			y = y0.			
		\end{cases}
		\label{eq:sys3}
	\end{equation}
	
	\begin{equation}
		\begin{cases}
			(x - c1)^2 + (y - c2)^2 = r^2,\\ 
			y = y1.				
		\end{cases}
		\label{eq:sys4}
	\end{equation}
	
	\begin{equation}
		\begin{gathered}
			db = r^2 - (y0 - c2)^2,\\
			dt = r^2 - (y1 - c2)^2.
			\label{eq:diskriminant2}
		\end{gathered}
	\end{equation}
	{\bfseries Note:} we define the default intersection points for the algorithm to work correctly as:
	
	\begin{equation}
		\begin{gathered}
			b1 = t1 =  \infty,\\
			b2 = t2 = -\infty.
			\label{eq:btinf}
		\end{gathered}
	\end{equation}
	
	We check the sign of $db, dt$ and find the intersection points:
	
	\begin{equation}
		\begin{cases}
			db > 0,\\ 
			b1 = c1 - \sqrt {db},\\
			b2 = c1 + \sqrt {db}.
			\label{eq:b1b2}
		\end{cases}
	\end{equation}
	
	\begin{equation}
		\begin{cases}
			dt > 0,\\ 
			t1 = c1 - \sqrt {dt},\\
			t2 = c1 + \sqrt {dt}.
			\label{eq:t1t2}
		\end{cases}
	\end{equation}
	
	We define the characteristics of rectangle as:
	
	\begin{equation}
		\begin{gathered}
			x0 = \max\{x0, \min\{b1, t1\}\},\\
			x1 = \min\{x1, \max\{b2, t2\}\}.
		\end{gathered}
	\end{equation}
	
	\item If all the values $dl$, $dr$, $db$, $dt$  are non-positive,  the rectangle has no intersections with the circle and we define the characteristics of rectangle as: 
	\begin{equation}
		x0 = x1.
		\label{eq:empty}
	\end{equation}
	
\end{itemize}	
\subsubsection*{Output}

Once all the parameters are updated we form the required rectangle {\bfseries rect\_approx} as:
\begin{equation}
	rect\_approx = Rect(x0, y0, x1, y1);
	\label{rect}
\end{equation}


\newpage

\label{Difference}
\begin{center} 
	\section*{Rect Geom::difference(Rect rect, Disk disk)}
\end{center}
\parindent = 0pt
\subsection*{Description}

The function approximates a rectangle and a circle difference area by horizontal and vertical lines. Basing on the intersection points of these lines, we construct a rectangle with a minimum area, which contains the difference area of the rectangle and the circle.

If the difference is the empty set, the function returns the rectangle with parameters that correspond to the condition (\ref{eq:cond1}).

\subsection*{Input parameters:}

The input of this function consists of two parameters:

\begin{itemize}
	\item {\bfseries	rect} is the rectangle, the element of class {\bfseries Rect} with characteristics:
	\begin{itemize}
		\item {\bfseries rectx0,recty0} are coordinates of the bottom left corner;
		\item {\bfseries rectx1, recty1} are coordinates of the top right corner.
	\end{itemize}
	
	To access these characteristics we use the methods  {\bfseries get\_rectx0()}, {\bfseries get\_recty0()}, {\bfseries get\_rectx1()}, {\bfseries get\_recty1()} implemented in the class {\bfseries Rect}.
	
	\item {\bfseries disk}  is the circle, the element of class {\bfseries Disk} with characteristics:
	\begin{itemize}
		\item {\bfseries center1, center2}  is the center of the circle;
		\item {\bfseries radius}  is  the radius of the circle.
	\end{itemize}
	
	To access these characteristics we use the methods {\bfseries get\_center1()}, {\bfseries get\_center2()}, {\bfseries get\_radius()} implemented in the class {\bfseries Disk}.
\end{itemize}

\subsection*{Output parameters:}

\begin{itemize}
	\item The function returns new rectangle {\bfseries rect\_approx} (the element of class {\bfseries Rect}) with a minimum area, which contains the difference area of the rectangle {\bfseries rect} and the circle {\bfseries disk}. The rectangle is formed as a result of the intersection of horizontal and vertical lines that approximate the difference area of the rectangle and the circle.
	
	If the difference is the empty set, the function returns a rectangle {\bfseries rect\_approx} with parameters that correspond to the condition (\ref{eq:cond1}).
	
\end{itemize}

\subsection*{Algorithm:}

\subsubsection*{Preprocessing}

We define:

\begin{itemize}
	
	\item the parameters of the rectangle {\bfseries rect} (\ref{eq:paramrect}),
	
	\item the parameters of the circle {\bfseries disk} (\ref{eq:paramdisk}),
	
	\item the values $dl, dr, db, dt$ (\ref{eq:diskriminant1},\ref{eq:diskriminant2}).
	
\end{itemize}

\subsubsection*{Approximation}

We consider two directions and update the characteristics of rectangle:

\begin{itemize}
	
	\item \underline {horizontal direction(the characteristics $y0, y1$)}:
	
	We consider the points of intersection of the circle with straight lines $x = x0$ and $x = x1$. We update the characteristics $y0, y1$ if $dl$ and $dr$ (\ref{eq:diskriminant1}) is positive.
	
	{\bfseries Note:} we define the default intersection points for the algorithm to work correctly as (\ref{eq:lrinf}).
	
	We check the sign of $dl, dr$ and find the intersection points (\ref{eq:l1l2}) and (\ref{eq:r1r2}).
	
	We define the characteristics of rectangle as:
	
	\begin{equation}
		\begin{gathered}
			y0 = \max\{y0, \min\{l2, r2\}\},\\
			y1 = \min\{y1, \max\{l1, r1\}\}.
		\end{gathered}
	\end{equation}
	
	\item \underline {vertical direction (the characteristics $x0, x1$) }
	
	We consider the points of intersection of the circle with straight lines $y = y0$ and $y = y1$.  We update the characteristics $x0, x1$ if $db$ and  $dt$ (\ref{eq:diskriminant2}) is positive.
	
	{\bfseries Note:} we define the default intersection points for the algorithm to work correctly as (\ref{eq:btinf}).
	
	We check the sign of $db, dt$ and find the intersection points (\ref{eq:b1b2}) and (\ref{eq:t1t2}).
	
	We define the characteristics of rectangle as:
	
	\begin{equation}
		\begin{gathered}
			x0 = \max\{x0, \min\{b2, t2\}\},\\
			x1 = \min\{x1, \max\{b1, t1\}\}.
		\end{gathered}
	\end{equation}
	
\end{itemize}	


\subsubsection*{Output}

Once all the parameters are updated we form the required rectangle {\bfseries rect\_approx} as (\ref{rect}).


\label{Empty}
\begin{center} 
	\section*{\bfseries bool Geom::empty\_set(Rect rect)}
\end{center} 

\subsection*{Description}

The function checks the parameters of the rectangle. If the parameters are not correct, this rectangle is empty. 

\subsection*{Input parameters:}

\begin{itemize}
	\item {\bfseries	rect} is the rectangle, the element of class {\bfseries Rect} with characteristics:
	
	\begin{itemize}
		\item {\bfseries rectx0,recty0} are coordinates of the bottom left corner;
		\item {\bfseries rectx1, recty1} are coordinates of the top right corner.
	\end{itemize}
\end{itemize}

To access these characteristics we use the methods  {\bfseries get\_rectx0()}, {\bfseries get\_recty0()}, {\bfseries get\_rectx1()}, {\bfseries get\_recty1()}, implemented in the class {\bfseries Rect}.checks the parameters  of the rectangle.

\subsection*{Output parameters:}

The function returns a boolean value {\bfseries true} if the rectangle is empty, and {\bfseries false} if it is not empty.

\subsection*{Algorithm:}

If the parameters of the rectangle correspond to the condition (\ref{eq:cond1}) this rectangle is empty and the function returns a boolean value {\bfseries true}, else  {\bfseries false}.

\newpage

\label{algoFPOP}
\begin{center} 
	\section*{\bfseries void OP::algoFPOP(std::vector$<$double$>$ y1, std::vector$<$double$>$ y2, int type)}
\end{center} 

\subsection*{Description}

	The function implements the Functional Pruning Optimal Partitioning algorithm with two types of pruning: "intersection of sets"($type = 1$) and "difference of intersection and union of sets"($type = 2$) for bivariate time series. 

\subsection*{Input parameters:}

\begin{itemize}
	\item {\bfseries y1} is the vector of data1(a univariate time series).
	
	\item {\bfseries y2} is the vector of data2(a univariate time series).
	
	\item {\bfseries type} is the value defined the type of pruning (1 = intersection set, 2 = difference of intersection and union set).
\end{itemize}

\subsection*{Output parameters:}

The function forms vectors the minimum value of the cost function {\bfseries m} and {\bfseries last\_chpts}.

\subsection*{Algorithm:}

\subsubsection*{Preprocessing}

We define:

\begin{itemize}
	\item $n = y1.size()$.
	
	\item $sy12 = vect_sy12(y1, y2)$ are the sum vectors.
	
	\item $m\_new, m\_temp$ are the variables for finding the minimum value of cost function.
	
	\item $chpt\_new$ is an integer number for the minimum argument.

	\item $cost\_new$ is a cost function, an element of class {\bfseries Cost}.

	\item $list\_geom$ is a list of class {\bfseries Geom} elements. 

	\item $it\_list$ is an itetator for $list\_geom$. 

	\item $geom\_new$ is an element of class {\bfseries Geom}.
\end{itemize}

\subsubsection*{Processing}

 We define $geom\_new$ for the first point $(y^1_1, y^2_1)$ of the bivariate time series as (\ref{eq:firstgeom})  and we add this element to the list $list\_geom$.
 
 \begin{equation}
 	geom\_new = Geom(y1[0], y2[0], sqrt(penalty), 0, penalty).
 	\label{eq:firstgeom}
 \end{equation}
 
 For each $t = 1,..,n-1$ we do:
 
\begin{itemize}
	\item \underline {The first run: Search $m\_new$}
	\begin{itemize}
		\item \underline {For each element of the list $list\_geom$:} 
		
		\begin{itemize}
			\item We get the element of the list $list\_geom$  pointed to by the iterator $it\_list$ to the variable $geom\_new$.
			
			\item We put the cost function to the variable $cost\_new$ using the method {\bfseries get\_cost\_t()} of {\bfseries Geom} class. We find the minimum value of the cost function $m\_temp$ using the method {\bfseries get\_min()}. 
		\end{itemize}
	
		\item As $m\_new$ we select the minimum of all values $m\_temp$.
		
		\item We put the $label\_t$ that corresponds $m\_new$ in the variable $chpt\_new$  using the method {\bfseries get\_label\_t()} of {\bfseries Geom} class and put this value to the vector $last\_chpts$ by position $t$.
		
		\item At the same time, we add the value $m\_new$ to the vector $m$ by position $t$. 
	\end{itemize}

	\item \underline {The second run: Pruning}
	
	\begin{itemize}
		\item Type = 1: Pruning "intersection"
		
		\begin{itemize}
			\item We define:
		
			\begin{itemize}
				\item $disk\_new$ is an element of {\bfseries Disk} class.
			
				\item $rect\_new$ is an element of {\bfseries Rect} class.
			
				\item $geom\_update$ is an element of {\bfseries Geom} class.
			
				\item $r\_new$ is a value of the radius for $disk\_new$.	
			\end{itemize}
	
			\item \underline {For each element of the list $list\_geom$:} 
			
			\begin{itemize}
				\item We get the element of the list $list\_geom$ pointed to by the iterator $it\_list$ to the variable $geom\_new$ .
				
				\item We put the cost function to the variable $cost\_new$ using the method {\bfseries get\_cost\_t()} of {\bfseries Geom} class.
				
				\item We calculate the radius $r\_new$ as (\ref{eq:radius}) and we forms the new disk $disk\_new$ (\ref{eq:disk}).
				
				\begin{equation}
					r\_new = \sqrt{\frac{m\_new - cost\_new.get\_mi\_1\_p()}{cost\_new.get\_coef()} - cost\_new.get\_coef\_Var()}.
					\label{eq:radius}
				\end{equation}
				
				\begin{equation}
					disk\_new = Disk(cost\_new.get\_mu1(), cost\_new.get\_mu2(), r\_new).
					\label{eq:disk}
				\end{equation}
				
				\item We put $rect\_t$ to the variable $rect\_new$ using the method {\bfseries get\_rect\_t()} of {\bfseries Geom} class.
				
				\item We intersect the disk $disk\_new$  with the $rect\_new$ and forms new element $geom\_update$ of {\bfseries Geom} class as (\ref{eq:geomupdate}).
			
				\begin{equation}
					\begin{gathered}
					Rect res\_inters = geom\_new.intersection(geom\_new.get\_rect\_t(), disk\_new),\\
					geom\_update = Geom(geom\_new.get\_label\_t(), geom\_new.get\_cost\_t(), res\_inters).
					\label{eq:geomupdate}
					\end{gathered}
				\end{equation}
				
				\item Using the method $empty_set(geom\_update.get\_rect\_t())$ we check the correctness  $rect\_t$ of new element $geom\_update$. 
				
				\item If it is not empty, we replace the element of list $list\_geom$ pointed to by the iterator $it\_list$  with $geom\_update$, otherwise we delete this element of list $list\_geom$.
			\end{itemize}			
		\end{itemize}
		
		\item Type = 2: Pruning "difference of intersection and union set"
		
		...................		
	\end{itemize}
\end{itemize}
\subsubsection*{Output:}

We have the vectors  {\bfseries m} and {\bfseries last\_chpts}.
	
\end{document}