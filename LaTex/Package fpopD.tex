\documentclass{report}
\usepackage[left=3cm,right=3cm, top=3cm,bottom=3cm,bindingoffset=0cm]{geometry}
\usepackage{amsmath}
\usepackage{hyperref}
\begin{document}
	\begin{center} \section*{Package fpop2D}\end{center}

	\parindent = 0pt
	\subsection*{Description}
	
	{\bfseries fpop2D} is an R package developed in Rcpp/C++ performing parametric changepoint detection in bivariate time series with  the Gaussian cost function. Changepoint detection uses the Functional Pruning Optimal Partitioning method (FPOP) based on an exact dynamic programming algorithm.
	
	The package implements FPOP method with two types of pruning: "intersection of sets" and "difference of intersection and union of sets". Also, the package implements the Pruned Exact Linear Time(PELT) method that is a direct part of the FPOP method.
	
	\subsection*{Package structure}
	\begin{itemize}
		\item  Part R
		\begin{itemize}
			\item \underline {fpop2D.R}
			
			The file contains the implementation of the following functions:
			
			 {\bfseries data\_gen2D} - generation of data of dimension 2 with a given values of means and changepoints
			 
			 {\bfseries plotFPOP2D} - plot of data with a  values of means and changepoints.
			 
			\item \underline {RcppExports.R} 
			
			The file contains the implementation of the function {\bfseries FPOP2D} that calls the function {\bfseries FPOP2D} implemented in C++.
			
		\end{itemize}
		\item Part C++
		\begin{itemize}
			\item \underline {Cost.h, Cost.cpp} 
			
			The files contain the implementation code for the class\hyperref [Cost]{\bfseries Cost}. 	
			
			\item \underline{Disk.h, Disk.cpp}
			
			The files contain the implementation code for the class \hyperref [Disk]{\bfseries Disk}. 
			
			\item  \underline{Rect.h, Rect.cpp}
			
			The files contain the implementation code for the class\hyperref [Rect] {\bfseries Rect}.
			 
			\item \underline{Geom.h, Geom.cpp} 
			
			The files contain the implementation code for the class \hyperref [Geom]{\bfseries Geom}. 
			
			\item \underline{OP.h, OP.cpp}
			
			The files contain the implementation code for the class \hyperref [OP]{\bfseries OP}. 
			
			\item \underline{fpop2D.cpp}
			
			The file contains the code of the function {\bfseries FPOP2D} that implements the change-point detection in bivariate time-series using the Functional Pruning Optimal Partitioning algorithm.
			
			\item \underline{RcppExports.cpp} 
			
			The file contains the code that exports data R/C ++ .	
		\end{itemize}
	\end{itemize}

\newpage
	
	We consider $(y_1, y_2)\in(R^2)^n$ - bivariate time-series when $y_1 = (y_1^1,..,y_n^1)$ and  $y_2 = (y_1^2,..,y_n^2)$ - two vectors of univariate data size $n$.
	
	We use the Gaussian cost of the segmented bivariate data when $m_{t}$ is the value of the optimal cost, $m_{0} = -\beta $. The cost function has the form: 
	
	\begin{equation}
		q_t^i( \theta_1,\theta_2 ) = 	m_{i-1} + \beta + (t-i+1)(( \theta_1 - E[y^1_{i:t}])^2 + ( \theta_2 - E[y^2_{i:t}])^2) + (t-i+1)(V(y^1_{i:t}) + V(y^2_{i:t}) ) , \hspace*{5mm} i = 1:t.
	\label{eq:cost}
	\end{equation}
	
	\begin{equation}
		E[y_{i:t}^k] = \frac{1}{t - i + 1}\sum_{l=i}^{t}(y_{i:t}^k) ,\hspace*{5mm} k = 1,2.
	\label{eq:mean}
	\end{equation}
	
	\begin{equation}
		V(y_{i:t}^k) = \frac{1}{t - i + 1}\sum_{l=i}^{t}(y_{i:t}^k)^2 - \left(\frac{1}{t - i + 1}\sum_{l=i}^{t}(y_{i:t}^k)\right)^2 ,\hspace*{5mm} k = 1,2.
	\label{eq:var}
	\end{equation}
	
	In the case of a Gaussian cost $q_t^i(\theta_1,\theta_2)$ is a paraboloid so that: 
	\begin{equation}
		m_t = min_{i = 1:t} \left\{m_{i-1} + (t-i+1)(V(y_{i:t}^1) + V(y_{i:t}^2)) + \beta  \right\}.
	\label{eq:min}
	\end{equation}
	
	We introduce the notations:
	
	\begin{equation}
		\begin{gathered}
			mu1 = E[y_{i:t}^1],\\
			mu2 = E[y_{i:t}^2],\\
			coef = (t - i + 1),\\
			mi\_1\_p = m_{i-1} + \beta,\\
			coef\_Var = (t-i+1)(V(y_{i:t}^1) + V(y_{i:t}^2)).
		\end{gathered}
		\label{eq:coef}
	\end{equation}
	
	Then the cost function  takes the form:
	
		\begin{equation}
	q_t^i(\theta_1,\theta_2) = mi\_1\_p+ coef\cdot((\theta_1 - mu1)^2 + (\theta_2 -mu1)^2) + coef\_Var, \hspace*{5mm} i = 1:t.
	\label{eq:costcoef}
	\end{equation}
	
	\subsection*{Class  Cost}
	\label{Cost}
	
	We define the class characteristics (\ref{eq:coef}): 
	
	\begin{itemize}
		\item {\bfseries coef}, {\bfseries coef\_Var}, {\bfseries mi\_1\_p}, {\bfseries mu1}, {\bfseries mu2} 
	\end{itemize}
 
	The class implements constructor:
	\begin{itemize}
		
		\item {\bfseries Cost(unsigned int i, unsigned int t, double** vectS, double mi\_1pen)} is the cost function at the time $[i,t]$. 
	\end{itemize}
	
	We define the class methods:
	
	\begin{itemize}
		\item {\bfseries get\_coef()}, {\bfseries get\_coef\_Var()}, {\bfseries get\_mi\_1\_p()}, {\bfseries get\_mu1()}, {\bfseries get\_mu2()}
		
		We use these  methods to access the characteristics of the class. 
		
		\item {\bfseries get\_min()}
		
		We use this method to get the minimum value of the cost function (\ref{eq:min}).
	\end{itemize} 

\newpage	

	\subsection*{Class Disk}
	\label{Disk}
	
	A class element is a circle that is defined by the center coordinates and the radius.
	
	\begin{itemize}
		\item We define the class characteristics: 
		\begin{itemize}
			\item {\bfseries center1, center2}  is the center of the circle.
		
			\item {\bfseries radius}  is  the radius of the circle.
		\end{itemize}
	
		\item The class implements two constructors:
		\begin{itemize}
			\item {\bfseries Disk()} All characteristics are equal zero by default. 
		
			\item {\bfseries Disk(double c1, double c2, double r)}
		
			$(c1, c2)$ are the center coordinates and $r$ is  the radius of the circle. 
		\end{itemize}
	
		\item We define the class methods:
	
		\begin{itemize}
			\item {\bfseries get\_center1()}, {\bfseries get\_center2()}, {\bfseries get\_radius()}
			
			We use these  methods to access the class characteristics.
		\end{itemize}
	\end{itemize}
	
	\subsection*{Class Rect}
	\label{Rect}
	
	A class element is a rectangle that is defined by the coordinates of the lower left-hand corner and the upper-right hand corner.
		
	\begin{itemize}
		\item We define the class characteristics: 
		\begin{itemize}
			\item {\bfseries rectx0,recty0} are coordinates of the lower left-hand corner.
			
			\item {\bfseries rectx1, recty1} are coordinates of the upper-right hand corner.
		\end{itemize}
		
		\item The class implements two constructors:
		\begin{itemize}
			\item {\bfseries Rect()}:
			\begin{itemize}
					\item $rectx0 = recty0 = -\inf)$;
					\item $rectx1 = recty1 = \inf)$.
			\end{itemize}
			
			\item {\bfseries Rect(double x0, double y0, double x1, double y1)}
			\begin{itemize}
				\item $(x0, y0)$ is the the lower left-hand corner;
				\item $(x1, y1)$ is the upper-right hand corner. 
			\end{itemize}	
		\end{itemize}
		
		\item We define the class methods:
		
		\begin{itemize}
			\item {\bfseries get\_rectx0()}, {\bfseries get\_recty0()}, {\bfseries get\_rectx1()},  {\bfseries get\_recty1()}
			
			We use these  methods to access the class characteristics.
			
			\item {\bfseries min\_ab(double a, double b)}, {\bfseries max\_ab(double a, double b)}
			
			We use these  methods  to find the minimum and maximum of two numbers.
			
			\item \hyperref [Intersection]{\bfseries intersection(Disk disk)} 
			
			The function approximates a rectangle and a circle intersection area by horizontal and vertical lines. Basing on the intersection points of these lines, we construct a rectangle with a minimum area, which contains the intersection area of the rectangle and the circle.
			
			\item \hyperref [Difference]{\bfseries difference(Disk disk)} 
			
			The function approximates a rectangle and a circle difference area by horizontal and vertical lines. Basing on the intersection points of these lines, we construct a rectangle with a minimum area, which contains the difference area of the rectangle and the circle.
		\end{itemize}
	\end{itemize}

	\newpage
		
	\subsection*{Class Geom}
	\label{Geom}
	
	\begin{itemize}
		\item We define the class characteristics: 
		\begin{itemize}
			\item {\bfseries label\_t} is the time moment.
			
			\item {\bfseries rect\_t} is the rectangle, the element of class {\bfseries Rect}.
			
			\item {\bfseries disks\_t\_1} is the list of active circles for the moment $t-1$.
		\end{itemize}
		
		\item The class implements two constructors:
		\begin{itemize}
			\item {\bfseries Geom(double c1, double c2, double r, double t, std::list<unsigned int> l\_disk)}
			\begin{itemize}
				\item $label\_t = t$;
				\item $disks\_t\_1 = l\_disk$;
				\item $rect\_t = Rect(c1-r, c2-r, c1+r, c2+r)$.
			\end{itemize}
			
			\item {\bfseries Geom(double t, std::list<unsigned int> l\_disk)}
			\begin{itemize}
				\item $label\_t = t$;
				\item $disks\_t\_1 = l\_disk$;
				\item $rect\_t = Rect()$.
			\end{itemize}	
		\end{itemize}
		
		\item We define the class methods:
		
		\begin{itemize}
			\item {\bfseries get\_label\_t()}, {\bfseries get\_rect\_t()}, {\bfseries get\_disks\_t\_1()}
			
			\item {\bfseries bool empty\_set(Rect R)}
			
			The function checks the parameters of the rectangle {\bfseries R}. If the parameters are not correct, this rectangle is empty. 
			
			\item {\bfseries intersection\_rect\_disk(Disk disk)} 
			
			The function calls the function {\bfseries intersection(Disk disk)}  implemented in the Rect class.
			
			\item {\bfseries difference\_rect\_disk(Disk disk)} 
			
			The function calls the function {\bfseries difference(Disk disk)}  implemented in the Rect class.
				
		\end{itemize}
	\end{itemize}

	\newpage	
	\subsection*{Class OP}
	\label{OP}
	
	\begin{itemize}
		\item We define the class characteristics: 
		
		\begin{itemize}
			\item {\bfseries n} is the number of data points.
			
			\item {\bfseries penalty} is a value of penalty (a non-negative real number).
			
			\item {\bfseries sy12} are sum vectors $\sum_{k=1}^{t} y^1_k$,  $\sum_{k=1}^{t}y^2_k$, $\sum_{k=1}^{t} (y^1_k)^2$,  $\sum_{k=1}^{t} (y^2_k)^2$, $t = 1:n$ .
			
			\item {\bfseries m} is the vector of the optimal cost value + $penalty$.
			
			\item {\bfseries changepoints} is the vector of changepoints.
			
			\item {\bfseries means1} is the vector of successive means for data1.
			
			\item {\bfseries means2} is the vector of successive means for data2.
			
			\item {\bfseries globalCost} is the global cost.
			
			\item {\bfseries geom\_activ} is the element of Geom class.
			
			\item {\bfseries list\_geom} is the list of Geom class element.
			
			\item {\bfseries it\_disk} is the iterator for {\bfseries list\_geom}.
			
			\item {\bfseries list\_disk} is the list of active labels for the moment $t-1$.
			
			\item {\bfseries it\_disk} is the iterator for {\bfseries list\_disk}.
		\end{itemize}
		
		\item The class implements the constructor:
		
		\begin{itemize}
			\item {\bfseries OP(std::vector$<$double$>$ y1, std::vector$<$double$>$ y2, double beta)}
			\begin{itemize}
				\item $penalty = beta$;
				
				\item $n = y1.size()$;
				
				\item we allocate memory for $sy12, m$.
			\end{itemize}	
		\end{itemize}
		
		\item We define the class methods:
		
		\begin{itemize}
			\item {\bfseries get\_changepoints()}, {\bfseries get\_means1()}, {\bfseries get\_means2()}, {\bfseries get\_globalCost()}, {\bfseries get\_n()}, {\bfseries get\_sy12()}, {\bfseries get\_geom\_activ()}, {\bfseries get\_it\_list()}  
			
			We use these  methods to access the class characteristics.
			
			\item {\bfseries vect\_sy12(std::vector$<$double$>$ y1, std::vector$<$double$>$ y2)}
			
			We use this method to find the sum vectors $\sum_{k=1}^{t} y^1_k$,  $\sum_{k=1}^{t}y^2_k
			$, $\sum_{k=1}^{t} (y^1_k)^2$,  $\sum_{k=1}^{t} (y^2_k)^2$, $t = 1:n$. 
			
			\item \hyperref [algoFPOP] {\bfseries algoFPOP(std::vector$<$double$>$ y1, std::vector$<$double$>$ y2, int type)} 
			
			The function implements the Functional Pruning Optimal Partitioning algorithm with two types of pruning: "intersection of sets"($type = 1$) and "difference of intersection and union of sets"($type = 2$). Also, the package implements the Pruned Exact Linear Time(PELT) method ($type = 0$).
		\end{itemize}
	\end{itemize}

\newpage

\label{Intersection}	
\begin{center} 
	\section*{void Rect::intersection(Disk disk)}
\end{center}

\parindent = 0pt
\subsection*{Description}

The function approximates a rectangle and a circle intersection area by horizontal and vertical lines. Basing on the intersection points of these lines, we construct a rectangle with a minimum area, which contains the intersection area of the rectangle and the circle.

If there is no intersection, the function makes the rectangle with parameters that correspond to the condition: 

\begin{equation}
	(rectx0 \ge rectx1) || (recty0 \ge recty1).
	\label{eq:cond1}
\end{equation}

\subsection*{Input parameters:}

The input of this function:

\begin{itemize}
	\item {\bfseries disk}  is the circle, the element of class {\bfseries Disk} with characteristics:
	\begin{itemize}
		\item {\bfseries center1, center2}  is the center of the circle;
		\item {\bfseries radius}  is  the radius of the circle.
	\end{itemize}
	
	To access these characteristics we use the methods {\bfseries get\_center1()}, {\bfseries get\_center2()}, {\bfseries get\_radius()} implemented in the class {\bfseries Disk}.
\end{itemize}

\subsection*{Algorithm:}

\subsubsection*{Preprocessing}

Using the methods {\bfseries get\_center1()}, {\bfseries get\_center2()}, {\bfseries get\_radius()} implemented in the class {\bfseries Disk} we define the parameters of the circle {\bfseries disk}:

\begin{equation}
	\begin{gathered}
		c1 = disk.get\_center1(),\\
		c2 = disk.get\_center2(),\\
		r = disk.get\_radius().
		\label{eq:paramdisk}
	\end{gathered}
\end{equation}

\subsubsection*{Approximation}

We consider two directions and update the characteristics of rectangle:

\begin{itemize}
	
	\item \underline {horizontal direction(the characteristics $recty0, recty1$)}:
	
	If   $rectx0 \le c1\le rectx1$ then
	\begin{equation}
		\begin{gathered}
			recty0 = \max\{recty0, c2-r\},\\
			recty1 = \min\{recty1, c2+r\}.
		\end{gathered}
	\end{equation}
	
	Otherwise, we consider the points of intersection of the circle with straight lines $x = rectx0$ and $x = rectx1$. The circle has two intersection points with a straight line if the discriminant is positive. We define the values $dl, dr$ (\ref{eq:diskriminant1}) as the value of a discriminant devided by  $4$ of each system (\ref{eq:sys1}, \ref{eq:sys2}):
	
	\begin{equation}
		\begin{cases}
			(x - c1)^2 + (y - c2)^2 = r^2,\\ 
			x = rectx0.
		\end{cases}
		\label{eq:sys1}
	\end{equation}
	
	\begin{equation}
		\begin{cases}
			(x - c1)^2 + (y - c2)^2 = r^2,\\ 
			x = rectx1.
		\end{cases}
		\label{eq:sys2}
	\end{equation}
	
	\begin{equation}
		\begin{gathered}
			dl = r^2 - (rectx0 - c1)^2,\\
			dr = r^2 - (rectx1 - c1)^2,\\
			\label{eq:diskriminant1}
		\end{gathered}
	\end{equation}
	
	{\bfseries Note:} we define the default intersection points for the algorithm to work correctly as:
	\begin{equation}
		\begin{gathered}
			l1 = r1 =  \infty,\\
			l2 = r2 = -\infty.
			\label{eq:lrinf}
		\end{gathered}
	\end{equation}
	
	We check the sign of $dl, dr$ and find the intersection points:
	
	\begin{equation}
		\begin{cases}
			dl > 0,\\ 
			l1 = c2 - \sqrt {dl},\\
			l2 = c2 + \sqrt {dl}.
			\label{eq:l1l2}
		\end{cases}
	\end{equation}
	
	\begin{equation}
		\begin{cases}
			dr > 0,\\ 
			r1 = c2 - \sqrt {dr},\\
			r2 = c2 + \sqrt {dr}.
			\label{eq:r1r2}
		\end{cases}
	\end{equation}
	
	We define the characteristics of rectangle as:
	
	\begin{equation}
		\begin{gathered}
			recty0 = \max\{recty0, \min\{l1, r1\}\},\\
			recty1 = \min\{recty1, \max\{l2, r2\}\}.
		\end{gathered}
	\end{equation}
	
	\item \underline {vertical direction (the characteristics $rectx0, rectx1$) }
	
	If   $recty0 \le c2 \le recty1$ then
	
	\begin{equation}
		\begin{gathered}
			rectx0 = \max\{rectx0, c1-r\},\\
			rectx1 = \min\{rectx1, c1+r\}.
		\end{gathered}
	\end{equation}
	
	Otherwise, we consider the points of intersection of the circle with straight lines $y = recty0$ and $y = recty1$. The circle has two intersection points with a straight line if the discriminant is positive. We define the values $db, dt$ (\ref{eq:diskriminant2}) as the value of a discriminant devided by  $4$ of each system (\ref{eq:sys3}, \ref{eq:sys4}):
	
	\begin{equation}
		\begin{cases}
			(x - c1)^2 + (y - c2)^2 = r^2,\\ 
			y = recty0.			
		\end{cases}
		\label{eq:sys3}
	\end{equation}
	
	\begin{equation}
		\begin{cases}
			(x - c1)^2 + (y - c2)^2 = r^2,\\ 
			y = recty1.				
		\end{cases}
		\label{eq:sys4}
	\end{equation}
	
	\begin{equation}
		\begin{gathered}
			db = r^2 - (recty0 - c2)^2,\\
			dt = r^2 - (recty1 - c2)^2.
			\label{eq:diskriminant2}
		\end{gathered}
	\end{equation}
	{\bfseries Note:} we define the default intersection points for the algorithm to work correctly as:
	
	\begin{equation}
		\begin{gathered}
			b1 = t1 =  \infty,\\
			b2 = t2 = -\infty.
			\label{eq:btinf}
		\end{gathered}
	\end{equation}
	
	We check the sign of $db, dt$ and find the intersection points:
	
	\begin{equation}
		\begin{cases}
			db > 0,\\ 
			b1 = c1 - \sqrt {db},\\
			b2 = c1 + \sqrt {db}.
			\label{eq:b1b2}
		\end{cases}
	\end{equation}
	
	\begin{equation}
		\begin{cases}
			dt > 0,\\ 
			t1 = c1 - \sqrt {dt},\\
			t2 = c1 + \sqrt {dt}.
			\label{eq:t1t2}
		\end{cases}
	\end{equation}
	
	We define the characteristics of rectangle as:
	
	\begin{equation}
		\begin{gathered}
			rectx0 = \max\{rectx0, \min\{b1, t1\}\},\\
			rectx1 = \min\{rectx1, \max\{b2, t2\}\}.
		\end{gathered}
	\end{equation}
	
	\item If all the values $dl$, $dr$, $db$, $dt$  are non-positive,  the rectangle has no intersections with the circle and we define the characteristics of rectangle as: 
	\begin{equation}
		rectx0 = rectx1.
		\label{eq:empty}
	\end{equation}
	
\end{itemize}	

\label{Empty}
\begin{center} 
	\section*{\bfseries bool Geom::empty\_set(Rect rect)}
\end{center} 

\subsection*{Description}

The function checks the parameters of the rectangle. If the parameters are not correct, this rectangle is empty. 

\subsection*{Input parameters:}

\begin{itemize}
	\item {\bfseries	rect} is the rectangle, the element of class {\bfseries Rect} with characteristics:
	
	\begin{itemize}
		\item {\bfseries rectx0,recty0} are coordinates of the bottom left corner;
		\item {\bfseries rectx1, recty1} are coordinates of the top right corner.
	\end{itemize}
\end{itemize}

To access these characteristics we use the methods  {\bfseries get\_rectx0()}, {\bfseries get\_recty0()}, {\bfseries get\_rectx1()}, {\bfseries get\_recty1()}, implemented in the class {\bfseries Rect}.checks the parameters  of the rectangle.

\subsection*{Output parameters:}

The function returns a boolean value {\bfseries true} if the rectangle is empty, and {\bfseries false} if it is not empty.

\subsection*{Algorithm:}

If the parameters of the rectangle correspond to the condition (\ref{eq:cond1}) this rectangle is empty and the function returns a boolean value {\bfseries true}, else  {\bfseries false}.

\newpage

\label{Difference}
\begin{center} 
	\section*{void Rect::difference(Disk disk)}
\end{center}
\parindent = 0pt
\subsection*{Description}

The function approximates a rectangle and a circle difference area by horizontal and vertical lines. Basing on the intersection points of these lines, we construct a rectangle with a minimum area, which contains the difference area of the rectangle and the circle.

If the difference is the empty set, the function makes the rectangle with parameters that correspond to the condition (\ref{eq:cond1}).

\subsection*{Input parameters:}

The input of this function consists:

\begin{itemize}	
	\item {\bfseries disk}  is the circle, the element of class {\bfseries Disk} with characteristics:
	\begin{itemize}
		\item {\bfseries center1, center2}  is the center and {\bfseries radius}  is  the radius of the circle.
	\end{itemize}
	
	To access these characteristics we use the methods {\bfseries get\_center1()}, {\bfseries get\_center2()}, {\bfseries get\_radius()} implemented in the class {\bfseries Disk}.
\end{itemize}

\subsection*{Algorithm:}

\subsubsection*{Preprocessing}

We define:

\begin{itemize}
	\item the parameters of the circle {\bfseries disk} (\ref{eq:paramdisk}),
	
	\item the values $dl, dr, db, dt$ (\ref{eq:diskriminant1},\ref{eq:diskriminant2}).	
\end{itemize}

\subsubsection*{Approximation}

We consider two directions and update the characteristics of rectangle:

\begin{itemize}
	
	\item \underline {horizontal direction(the characteristics $recty0, recty1$)}:
	
	We consider the points of intersection of the circle with straight lines $x = rectx0$ and $x = rectx1$. We update the characteristics $recty0, recty1$ if $dl$ and $dr$ (\ref{eq:diskriminant1}) is positive.
	
	{\bfseries Note:} we define the default intersection points for the algorithm to work correctly as (\ref{eq:lrinf}).
	
	We check the sign of $dl, dr$ and find the intersection points (\ref{eq:l1l2}) and (\ref{eq:r1r2}).
	
	We define the characteristics of rectangle as:
	
	\begin{equation}
		\begin{gathered}
			recty0 = \max\{recty0, \min\{l2, r2\}\},\\
			recty1 = \min\{recty1, \max\{l1, r1\}\}.
		\end{gathered}
	\end{equation}
	
	\item \underline {vertical direction (the characteristics $rectx0, rectx1$) }
	
	We consider the points of intersection of the circle with straight lines $y = recty0$ and $y = recty1$.  We update the characteristics $rectx0, rectx1$ if $db$ and  $dt$ (\ref{eq:diskriminant2}) is positive.
	
	{\bfseries Note:} we define the default intersection points for the algorithm to work correctly as (\ref{eq:btinf}).
	
	We check the sign of $db, dt$ and find the intersection points (\ref{eq:b1b2}) and (\ref{eq:t1t2}).
	
	We define the characteristics of rectangle as:
	
	\begin{equation}
		\begin{gathered}
			rectx0 = \max\{rectx0, \min\{b2, t2\}\},\\
			rectx1 = \min\{rectx1, \max\{b1, t1\}\}.
		\end{gathered}
	\end{equation}
	
\end{itemize}	


\newpage

\label{algoFPOP}
\begin{center} 
	\section*{\bfseries void OP::algoFPOP(std::vector$<$double$>$ y1, std::vector$<$double$>$ y2, int type)}
\end{center} 

\subsection*{Description}

	The function implements FPOP method with two types of pruning: "intersection of sets"($type = 1$) and "difference of intersection and union of sets"($type = 2$) for bivariate time series. Also, the function implements PELT method ($type = 0$).

\subsection*{Input parameters:}

\begin{itemize}
	\item {\bfseries y1} is the vector of data1 (a univariate time series).
	\item {\bfseries y2} is the vector of data2 (a univariate time series).
	\item {\bfseries type} is the value defined the type of pruning ( 0 = PELT, 1 = FPOP(intersection of sets), 2 = FPOP (difference of intersection and union sets)).
\end{itemize}

\subsection*{Output parameters:}

The function forms the vectors {\bfseries changepoints}, {\bfseries means1, means2} and the value {\bfseries globalCost}.

\subsection*{Algorithm:}

\subsubsection*{Preprocessing}

\begin{itemize}
	\item We define:
	\begin{itemize}
		\item $n = y1.size()$.
		\item $sy12 = vect\_sy12(y1, y2)$.	
		\item $m[0] = 0$.
	\end{itemize}	
	\item We allocate memory for the matrix $last\_chpt\_mean$:
	\begin{itemize}
		\item the vector of best last changepoints.
		\item the vector of means for best last changepoints $y1$.
		\item the vector of means for best last changepoints $y2$.
	\end{itemize}	
\end{itemize}

\subsubsection*{Processing}

For each $t = 0,..,n-1$ we do:
 
\begin{itemize}
	\item We define:
	\begin{itemize}
		\item $min\_val = \inf$  is a value for searching  of the cost function minimum.
		\item $lbl$ is a best last position for $t$.
		\item $mean1$ is a value of mean of the interval $(lbl, t)$ for $y1$.
		\item $mean2$ is a value of mean of the interval $(lbl, t)$ for $y2$.
	\end{itemize}
	\item \underline {New $D_t^t$}

	 We create $geom\_activ$ for the  point $(y^1_t, y^2_t)$ of the bivariate time series as (\ref{eq:firstgeom})  and we add this element to the list $list\_geom$.
	
	\begin{equation}
		geom\_activ = Geom(t, list\_disk).
		\label{eq:firstgeom}
	\end{equation}	
	\item We define $list\_activ\_disk$ is a list of active disks for $t$.	
	\item \underline {The first run: Searching of $m[t+1]$}
	
	For each element of the list $list\_geom$ we do:
		
	\begin{itemize}
		\item We define $u$ is the $label\_t$ of the current element.
		\item We create $cost\_activ$  is the cost function for the interval $(u,t)$.	
		\item We find the minimum value  $min\_val$ of the cost function using the method {\bfseries get\_min()}. 
	\end{itemize}
	\item We choose the minimum  among all found values ${min\_val}$ and we define the value $lbl$ that correspond this minimum.	
	\item We put the value $\min{min\_val} + penalty$ to the vector $m$  by the position $t+1$ .	
	\item We put the values  $lbl$, $mean1$ and $mean2$ that corresponds $m[t+1]$ to the matrix $last\_chpts$.
	\item \underline {The second run: Pruning}
	
	For each element of the list $list\_geom$ we do:
	
	\begin{itemize}
		\item We define the current element and his $label\_t$ as $geom\_activ$ and $lbl$ .
		\item We define and calculate the cost function for the interval $(lbl,t)$ as \ref{eq:CostInter} and $r2\_inter$ is the radius to the second power of the disk $D_{lbl}^t$ as \ref{eq:radiusInter}.
		
		\begin{equation}
		cost\_inter = Cost(lbl, t, sy12[lbl],sy12[t+1], m[lbl]);
			\label{eq:CostInter}
		\end{equation}
		
		\begin{equation}
			r2\_inter = \frac {m[t + 1] - m[lbl] - cost\_inter.get\_coef\_Var()}{   cost\_inter.get\_coef()}.
			\label{eq:radiusInter}
		\end{equation}
		\item If $type \ge 0$:
		
		\underline {PELT-pruning:}
		
		 If $r2\_inter \le 0$ we delete this element of list $list\_geom$.
		\
		\item If $type \ge 1$ and  $r2\_inter > 0$:
		
		We define $disk\_inter = Disk(cost\_inter.get\_mu1(), cost\_inter.get\_mu2(), \sqrt{r2\_inter})$.
		
		We update the rectangle $rect\_t$ of $geom\_activ$ using the function $intersection\_geom\_disk(disk\_inter)$.
		
		\underline {FPOP-pruning(intersection of sets):} 
	 	\begin{itemize}
		 	\item If new rectangle $rect\_t$ is empty we delete this element of list $list\_geom$. 	
		 	\item Else we add  $label\_t$ of $geom\_activ$ to the list $list\_activ\_disk$.
		\end{itemize}
		\item If $type = 2$ and $rect\_t$ is not empty:
			
		We define the list $list\_disk$ as the list $disks\_t\_1$ of $geom\_activ$.
				
		For each element of  $list\_disk$:
		\begin{itemize} 
			\item We define his value $*it\_disk$.
			\item We define and calculate the cost function for the interval $(*it\_disk, lbl-1)$ as \ref{eq:CostDif} and $r2\_dif$ is the radius to the second power of the disk $D^{lbl-1}_{*it\_disk}$ as \ref{eq:radiusDif}.	
			\begin{equation}
					cost\_dif = Cost(*it\_disk, lbl-1, sy12[*it\_disk],sy12[lbl], m[*it\_disk]);
					\label{eq:CostDif}
			\end{equation}
			
			\begin{equation}
					r2\_dif = \frac {m[lbl] - m[*it\_disk] - cost\_dif.get\_coef\_Var()}{   			cost\_dif.get\_coef()}.
						\label{eq:radiusDif}
			\end{equation}
			\item If $r2\_dif > 0$: 
		
			We define $disk\_dif = Disk(cost\_dif.get\_mu1(), cost\_dif.get\_mu2(), \sqrt{r2\_dif})$.
			
			We update the rectangle $rect\_t$ of $geom\_activ$ using the function $difference\_geom\_disk(disk\_dif)$.
			
			\underline {FPOP-pruning (difference of intersection and union of sets):}
					
			If new rectangle $rect\_t$ is empty we delete this element of list $list\_geom$ and we finish to treat the $list\_disk$ elements.
					
		\end{itemize} 
	\item We define the elements of list $list\_disk$ as $list\_activ\_disk$. We will use this list $list\_disk$  as the value of list $disks\_t\_1$ to generate a new element of list $list\_geom$ in the next iteration.
	\end{itemize} 
\end{itemize} 
	
\subsubsection*{Output:}

Knowing the values of the matrix $last\_chpts$ and vector  $m$ we forme the vectors  $changepoints$, $means1$, $means2$ and the value $globalCost$.
	
\end{document}