\documentclass{report}
\usepackage[left=2cm,right=2cm, top=2cm,bottom=2cm,bindingoffset=0cm]{geometry}
\usepackage{amsmath}
\begin{document}
	
	\begin{center} 
		\section*{Rect Geom::intersection(Rect rect, Disk disk)}
	\end{center}
	\parindent = 0pt
	\subsection*{Description}
	
	The function approximates a rectangle and a circle intersection area by horizontal and vertical lines. Basing on the intersection points of these lines, we construct a rectangle with a minimum area, which contains the intersection area of the rectangle and the circle.
	
	If there is no intersection, the function returns the rectangle with parameters that correspond to the condition: 
	
	\begin{equation}
		(rectx0 \ge rectx1) || (recty0 \ge recty1).
		\label{eq:cond1}
	\end{equation}
	
	\subsection*{Input parameters:}
	
	The input of this function consists of two parameters:
	
	\begin{itemize}
		\item {\bfseries	rect} is the rectangle, the element of class {\bfseries Rect} with characteristics:
		\begin{itemize}
			\item {\bfseries rectx0,recty0} are coordinates of the bottom left corner;
			\item {\bfseries rectx1, recty1} are coordinates of the top right corner.
		\end{itemize}
	
		To access these characteristics we use the methods  {\bfseries get\_rectx0()}, {\bfseries get\_recty0()}, {\bfseries get\_rectx1()}, {\bfseries get\_recty1()} implemented in the class {\bfseries Rect}.
		
		\item {\bfseries disk}  is the circle, the element of class {\bfseries Disk} with characteristics:
		\begin{itemize}
			\item {\bfseries center1, center2}  is the center of the circle;
			\item {\bfseries radius}  is  the radius of the circle.
		\end{itemize}
		
		To access these characteristics we use the methods {\bfseries get\_center1()}, {\bfseries get\_center2()}, {\bfseries get\_radius()} implemented in the class {\bfseries Disk}.
	\end{itemize}
	
	\subsection*{Output parameters:}
	
	\begin{itemize}
		\item The function returns new rectangle {\bfseries rect\_approx} (the element of class {\bfseries Rect}) with a minimum area, which contains the intersection area of the rectangle {\bfseries rect} and the circle {\bfseries disk}. The rectangle is formed as a result of the intersection of horizontal and vertical lines that approximate the intersection area of the rectangle and the circle.
		
		If there is no intersection, the function returns a rectangle {\bfseries rect\_approx} with parameters that correspond to the condition (\ref{eq:cond1}).

	\end{itemize}

	\subsection*{Algorithm:}
	
	\subsubsection*{Preprocessing}
	
	Using the methods  {\bfseries get\_rectx0()}, {\bfseries get\_recty0()}, {\bfseries get\_rectx1()}, {\bfseries get\_recty1()} implemented in the class {\bfseries Rect} we define the parameters of the rectangle {\bfseries rect}:
	
	\begin{equation}
		\begin{gathered}	
			x0 = rect.get\_rectx0(),\\
			x1 = rect.get\_rectx1(),\\
			y0 = rect.get\_recty0(),\\
			y1 = rect.get\_recty1().
			\label{eq:paramrect}
		\end{gathered}
	\end{equation}
	Using the methods {\bfseries get\_center1()}, {\bfseries get\_center2()}, {\bfseries get\_radius()} implemented in the class {\bfseries Disk} we define the parameters of the circle {\bfseries disk}:
		
	\begin{equation}
		\begin{gathered}
			c1 = disk.get\_center1(),\\
			c2 = disk.get\_center2(),\\
			r = disk.get\_radius().
			\label{eq:paramdisk}
		\end{gathered}
	\end{equation}

	\subsubsection*{Approximation}
	
	We consider one the following cases:
	\begin{itemize}
		\item \underline{The center of the circle is inside the rectangle.}
				
		 If the center of the disk inside the rectangle, we define the characteristics of rectangle as:
		\begin{equation}
			\begin{gathered}
				x0 = \max\{x0, c1-r\},\\
				x1 = \min\{x1, c1+r\},\\
				y0 = \max\{y0, c2-r\},\\
				y1 = \min\{y1, c2+r\}.
			\end{gathered}
		\end{equation}
	
		\item \underline{The center of the circle is outside the rectangle.}
		
		 If the center of the circle is outside the rectangle, we calculate the values of the discriminants of quadratic equations obtained from the systems (\ref{eq:sys1} - \ref{eq:sys4}). These systems write: 
	
		\begin{equation}
			\begin{cases}
			(x - c1)^2 + (y - c2)^2 = r^2,\\ 
			x = x0.
			\end{cases}
		\label{eq:sys1}
		\end{equation}
	
		\begin{equation}
			\begin{cases}
				(x - c1)^2 + (y - c2)^2 = r^2,\\ 
				x = x1.
			\end{cases}
		\label{eq:sys2}
		\end{equation}
	
		\begin{equation}
			\begin{cases}
				(x - c1)^2 + (y - c2)^2 = r^2,\\ 
				y = y0.			
			\end{cases}
		\label{eq:sys3}
		\end{equation}
	
		\begin{equation}
			\begin{cases}
				(x - c1)^2 + (y - c2)^2 = r^2,\\ 
	 			y = y1.				
			\end{cases}
		\label{eq:sys4}
		\end{equation}
	
		We define the values $dl, dr, db, dt$ as the value of a discriminant devided by  $4$ of each system (\ref{eq:sys1} - \ref{eq:sys4}):
		 
		 \begin{equation}
		 	\begin{gathered}
		 		dl = r^2 + (x0 - c1)^2,\\
		 		dr = r^2 + (x1 - c1)^2,\\
		 		db = r^2 + (y0 - c2)^2,\\
		 		dt = r^2 + (y1 - c2)^2.
		 		\label{eq:diskriminant}
		 	\end{gathered}
		 \end{equation}
		
		If all the values $dl$, $dr$, $db$, $dt$  are non-positive,  the rectangle has no intersections with the circle and we define the characteristics of rectangle as: 
		
		\begin{equation}
			 x0 = x1.
			 \label{eq:empty}
		\end{equation}
	
		Otherwise, we consider two directions and update the characteristics of rectangle:
		
		\begin{itemize}
			
			\item \underline {horizontal direction(the characteristics $y0, y1$)}:
			
			If   $x0 \le c1\le x1$ then
			\begin{equation}
				\begin{gathered}
					y0 = \max\{y0, c2-r\},\\
					y1 = \min\{y1, c2+r\}.
				\end{gathered}
			\end{equation}
			Otherwise, we consider the points of intersection of the circle with straight lines $x = x0$ and $x = x1$. The circle has two intersection points with a straight line if the discriminant is positive.
			
			{\bfseries Note:} we define the default intersection points for the algorithm to work correctly as:
			\begin{equation}
				\begin{gathered}
				l1 = r1 =  \infty,\\
				l2 = r2 = -\infty.
				\label{eq:lrinf}
				\end{gathered}
			\end{equation}
		
			We check the sign of $dl, dr$ and find the intersection points:
			
			\begin{equation}
				\begin{cases}
			     	dl > 0,\\ 
					l1 = c2 - \sqrt {dl},\\
					l2 = c2 + \sqrt {dl}.
					\label{eq:l1l2}
				\end{cases}
		 	\end{equation}
	 	
			\begin{equation}
				\begin{cases}
					dr > 0,\\ 
					r1 = c2 - \sqrt {dr},\\
					r2 = c2 + \sqrt {dr}.
					\label{eq:r1r2}
				\end{cases}
			\end{equation}
			
			We define the characteristics of rectangle as:
			
			\begin{equation}
				\begin{gathered}
					y0 = \max\{y0, \min\{l1, r1\}\},\\
					y1 = \min\{y1, \max\{l2, r2\}\}.
				\end{gathered}
			\end{equation}
				
			\item \underline {vertical direction (the characteristics $x0, x1$) }
			
			If   $y0 \le c2 \le y1$ then
			
			\begin{equation}
				\begin{gathered}
					x0 = \max\{x0, c1-r\},\\
					x1 = \min\{x1, c1+r\}.
				\end{gathered}
		\end{equation}
	
			Otherwise, we consider the points of intersection of the circle with straight lines $y = y0$ and $y = y1$. The circle has two intersection points with a straight line if the discriminant is positive.
			
			{\bfseries Note:} we define the default intersection points for the algorithm to work correctly as:
			
			\begin{equation}
				\begin{gathered}
					b1 = t1 =  \infty,\\
					b2 = t2 = -\infty.
					\label{eq:btinf}
			\end{gathered}
		\end{equation}
		
		We check the sign of $db, dt$ and find the intersection points:
			
		\begin{equation}
			\begin{cases}
				db > 0,\\ 
				b1 = c1 - \sqrt {db},\\
				b2 = c1 + \sqrt {db}.
				\label{eq:b1b2}
			\end{cases}
		\end{equation}
		
		\begin{equation}
			\begin{cases}
				dt > 0,\\ 
				t1 = c1 - \sqrt {dt},\\
				t2 = c1 + \sqrt {dt}.
				\label{eq:t1t2}
			\end{cases}
		\end{equation}
			
		We define the characteristics of rectangle as:
		
			\begin{equation}
				\begin{gathered}
					x0 = \max\{x0, \min\{b1, t1\}\},\\
					x1 = \min\{x1, \max\{b2, t2\}\}.
			\end{gathered}
		\end{equation}
	
		\end{itemize}	
	\end{itemize}

	\subsubsection*{Output}

Once all the parameters are updated we form the required rectangle {\bfseries rect\_approx} as:
\begin{equation}
	rect\_approx = Rect(x0, y0, x1, y1);
	\label{rect}
\end{equation}
\newpage
	
\begin{center} 
	\section*{Rect Geom::difference(Rect rect, Disk disk)}
\end{center}
\parindent = 0pt
\subsection*{Description}

The function approximates a rectangle and a circle difference area by horizontal and vertical lines. Basing on the intersection points of these lines, we construct a rectangle with a minimum area, which contains the difference area of the rectangle and the circle.

If the difference is the empty set, the function returns the rectangle with parameters that correspond to the condition (\ref{eq:cond1}).

\subsection*{Input parameters:}

The input of this function consists of two parameters:

\begin{itemize}
	\item {\bfseries	rect} is the rectangle, the element of class {\bfseries Rect} with characteristics:
	\begin{itemize}
		\item {\bfseries rectx0,recty0} are coordinates of the bottom left corner;
		\item {\bfseries rectx1, recty1} are coordinates of the top right corner.
	\end{itemize}
	
	To access these characteristics we use the methods  {\bfseries get\_rectx0()}, {\bfseries get\_recty0()}, {\bfseries get\_rectx1()}, {\bfseries get\_recty1()} implemented in the class {\bfseries Rect}.
	
	\item {\bfseries disk}  is the circle, the element of class {\bfseries Disk} with characteristics:
	\begin{itemize}
		\item {\bfseries center1, center2}  is the center of the circle;
		\item {\bfseries radius}  is  the radius of the circle.
	\end{itemize}
	
	To access these characteristics we use the methods {\bfseries get\_center1()}, {\bfseries get\_center2()}, {\bfseries get\_radius()} implemented in the class {\bfseries Disk}.
\end{itemize}

\subsection*{Output parameters:}

\begin{itemize}
	\item The function returns new rectangle {\bfseries rect\_approx} (the element of class {\bfseries Rect}) with a minimum area, which contains the difference area of the rectangle {\bfseries rect} and the circle {\bfseries disk}. The rectangle is formed as a result of the intersection of horizontal and vertical lines that approximate the difference area of the rectangle and the circle.
	
	If the difference is the empty set, the function returns a rectangle {\bfseries rect\_approx} with parameters that correspond to the condition (\ref{eq:cond1}).
	
\end{itemize}

\subsection*{Algorithm:}

\subsubsection*{Preprocessing}

 We define:
 
 \begin{itemize}
 	
 	\item the parameters of the rectangle {\bfseries rect} (\ref{eq:paramrect}),
 	
 	\item the parameters of the circle {\bfseries disk} (\ref{eq:paramdisk}),
 	
 	\item the values $dl, dr, db, dt$ (\ref{eq:diskriminant}).
 		
 \end{itemize}

\subsubsection*{Approximation}

We consider one the following cases:

\begin{itemize}
	
	\item The difference area is the empty set.
	
	\begin{itemize}
		
 		\item If all the values $dl$, $dr$, $db$, $dt$  are positive, the rectangle is inside the disk and the difference area is the empty set. We define the characteristics of rectangle as (\ref{eq:empty}).
 		
 	\end{itemize}
 
   \item The difference area is not the empty set.
 
   We consider two directions and update the characteristics of rectangle:
	
	\begin{itemize}
		
		\item \underline {horizontal direction(the characteristics $y0, y1$)}:
		
		We consider the points of intersection of the circle with straight lines $x = x0$ and $x = x1$. The circle has two intersection points with a straight line if the discriminant is positive.
		
		{\bfseries Note:} we define the default intersection points for the algorithm to work correctly as (\ref{eq:lrinf}).
		
		We check the sign of $dl, dr$ and find the intersection points (\ref{eq:l1l2}) and (\ref{eq:r1r2}).
		
		We define the characteristics of rectangle as:
		
		\begin{equation}
			\begin{gathered}
				y0 = \max\{y0, \min\{l2, r2\}\},\\
				y1 = \min\{y1, \max\{l1, r1\}\}.
			\end{gathered}
		\end{equation}
		
		\item \underline {vertical direction (the characteristics $x0, x1$) }
		
		 We consider the points of intersection of the circle with straight lines $y = y0$ and $y = y1$. The circle has two intersection points with a straight line if the discriminant is positive.
		
		{\bfseries Note:} we define the default intersection points for the algorithm to work correctly as(\ref{eq:btinf}).
		
		We check the sign of $db, dt$ and find the intersection points  (\ref{eq:b1b2}) and (\ref{eq:t1t2}).
		
		We define the characteristics of rectangle as:
		
		\begin{equation}
			\begin{gathered}
				x0 = \max\{x0, \min\{b2, t2\}\},\\
				x1 = \min\{x1, \max\{b1, t1\}\}.
			\end{gathered}
		\end{equation}
		
	\end{itemize}	
\end{itemize}

\subsubsection*{Output}

Once all the parameters are updated we form the required rectangle {\bfseries rect\_approx} as (\ref{rect}).

\newpage
\begin{center} 
	\section*{\bfseries bool Geom::empty\_set(Rect rect)}
\end{center} 

\subsection*{Description}

The function checks the parameters of the rectangle. If the parameters are not correct, this rectangle is empty. 

\subsection*{Input parameters:}

\begin{itemize}
	\item {\bfseries	rect} is the rectangle, the element of class {\bfseries Rect} with characteristics:
	
	\begin{itemize}
		\item {\bfseries rectx0,recty0} are coordinates of the bottom left corner;
		\item {\bfseries rectx1, recty1} are coordinates of the top right corner.
	\end{itemize}
\end{itemize}

To access these characteristics we use the methods  {\bfseries get\_rectx0()}, {\bfseries get\_recty0()}, {\bfseries get\_rectx1()}, {\bfseries get\_recty1()}, implemented in the class {\bfseries Rect}.checks the parameters  of the rectangle.

\subsection*{Output parameters:}

The function returns a boolean value {\bfseries true} if the rectangle is empty, and {\bfseries false} if it is not empty.

\subsection*{Algorithm:}

If the parameters of the rectangle correspond to the condition (\ref{eq:cond1}) this rectangle is empty and the function returns a boolean value {\bfseries true}, else  {\bfseries false}. 

\end{document}